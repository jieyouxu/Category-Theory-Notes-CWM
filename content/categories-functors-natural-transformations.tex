\section{Categories, Functors, and Natural Transformation}

\subsection{Categories Axioms}

\begin{definition}[Metagraph]
    A \textit{metagraph} has:
    \begin{itemize}
        \item Contents
        \begin{enumerate}
            \item \textbf{Objects} $a, b, c, \dots$.
            \item \textbf{Arrows} (or \textbf{morphisms}) $f, g, h, \dots$.
        \end{enumerate}
        \item Operations
        \begin{enumerate}
            \item \textbf{Domain}: each arrow $f$ is assigned an object $a = \Domain{f}$.
            \item \textbf{Codomain}: each arrow $f$ is assigned an object $b = \Codomain{f}$.
        \end{enumerate}
    \end{itemize}
    
    Each \textit{arrow} $f$ starts at its \textit{domain} (or \textit{source}) $a$ and ends at its \textit{codomain} (or \textit{target}) $b$, denoted $f \colon a \to b$ or graphically,
    \begin{figure}[H]
        \centering
        \begin{tikzcd}
            a \arrow[r, "f"] & b
        \end{tikzcd}
        \caption{Arrow $f$ starting at domain $a$ and ending at codomain $b$.}
        \label{fig:arrow}
    \end{figure}
    
    Here $\Domain{f} = a$ and $\Codomain{f} = b$.
\end{definition}

\begin{definition}[Metacategory]
    A \textit{metacategory} is a metagraph with two additional operations:
    \begin{enumerate}
        \item \textbf{Identity}: each object $a$ is assigned an \textit{identity arrow} $\Id_a \coloneqq 1_a \colon a \to a$.
        \begin{figure}[H]
            \centering
            \begin{tikzcd}
                a \arrow[loop right, "\Id_a = 1_a"]
            \end{tikzcd}
            \caption{Identity arrow $\Id_a$ of object $a$ with $\Domain{\Id_a} = \Codomain{\Id_a} = a$.}
            \label{fig:identity-arrow}
        \end{figure}
        \item \textbf{Composition}: each pair of arrows $\langle g, f \rangle$ with $\Domain{g} = \Codomain{f}$ is assigned a \textit{composite} arrow $g \circ f$, where
        \begin{equation}
            g \circ f \colon \Domain{f} \to \Codomain{g}
        \end{equation}
        \begin{figure}[H]
            \centering
            \begin{tikzcd}
                a \arrow[r, swap, "f"] \arrow[rr, bend left, "g \circ f"] & b \arrow[r, swap, "g"] & c
            \end{tikzcd}
            \caption{Composite arrow $g \circ f$ given $\Domain{g} = \Codomain{f}$.}
            \label{fig:composite-arrow}
        \end{figure}
    \end{enumerate}
    
    The two additional operations in a metacategory satisfies the two axioms:
    \begin{enumerate}
        \item \textbf{Associativity}: given any objects $a, b, c, d$ and arrows in between of the form
        \begin{figure}[H]
            \centering
            \begin{tikzcd}
                a \arrow[r, "f"] & b \arrow[r, "g"] & c \arrow[r, "h"] & d
            \end{tikzcd}
        \end{figure}
        Then the equality below is satisfied
        \begin{equation}
            h \circ (g \circ f) = (h \circ g) \circ f
        \end{equation}
        That is,
        \begin{figure}[H]
            \centering
            \begin{tikzcd}[column sep=8em, row sep=4em]
                a \arrow[r, "h \circ (g \circ f) = (h \circ g) \circ f"] \arrow[d, swap, "f"] \arrow[rd, near end, "g \circ f"] & d \\
                b \arrow[r, swap, "g"] \arrow[ru, near start, "h \circ g"] & c \arrow[u, swap, "h"]
            \end{tikzcd}
            \caption{Associativity axiom.}
            \label{fig:associativity-axiom}
        \end{figure}
        \item \textbf{Unit Law}: for any arrows $f \colon a \to b$ and $g \colon b \to c$, their respective composition with identity arrow $1_b$ gives
        \begin{align}
            1_b \circ f &= f \\
            g \circ 1_b &= g
        \end{align}
        That is,
        \begin{figure}[H]
            \centering
            \begin{tikzcd}[row sep=3em]
                a \arrow[r, "f"] \arrow[rd, "f"] & b \arrow[d, "1_b"] \arrow[rd, "g"] & \\
                  & b \arrow[r, "g"] & c
            \end{tikzcd}
            \caption{Unit Law.}
            \label{fig:unit-law}
        \end{figure}
    \end{enumerate}
\end{definition}

\begin{example}
    In the metacategory of \textit{sets}, with objects being all sets and arrows being all functions. A function in this context is a function with specified domain and codomain, i.e. a function $f \colon X \to Y$ maps from its domain $X$ to its codomain $Y$. A rule $x \mapsto f(x)$, i.e. the suitable set of pairs $\langle x, f(x) \rangle$, maps each element $x \in X$ to an element $f(x) \in Y$.
    
    For any set $S$, the mapping $\Forall s \in S \colon s \mapsto s$ is the \textit{identity function} $1_S \colon S \to S$.
    
    Should $S \subseteq Y$, then $\Forall s \colon s \in S \to s \in Y$ describes the \textit{inclusion function} $S \to Y$. The identity function and inclusion function are \textit{different}, unless $S = Y$ when both the identity and inclusion relation holds true, i.e. $S = Y \to S \subseteq Y$.
\end{example}

\begin{definition}[Commutative Diagram]
    A diagram is \textit{commutative} iff all paths that share a common starting point and ending point are equivalent \cite{commutative-diagrams}.
\end{definition}

\begin{remark}
    If $b$ is any object of a metacategory $C$, then its identity arrow $1_b$ is determined by the unit law. It is thus useful to denote the identity arrow $1_b$ with the object $b$ itself, with $b \colon b \to b$ and as a result $1_b = b = \Id_b$ are all aliases for each other.
\end{remark}

\begin{definition}[Arrows-Only Metacategory]
    Because objects of a metacategory have one-to-one correspondence with its identity arrows, the objects may be safely omitted and only arrows are to dealt with. Then an \textit{arrows-only metacategory} $C$ consists of the data:
    \begin{enumerate}
        \item \textbf{Arrows}.
        \item \textbf{Composable pairs of arrows}: $\langle g, f \rangle$.
        \item \textbf{Composite}: an operation assigning the composite $g \circ f$ to each composable pair $\langle g, f \rangle$.
    \end{enumerate}
    
    Then $g \circ f$ is \textit{defined} iff $\langle g, f \rangle$ is a composable pair.
    
    In the arrows-only metacategory $C$, an identity arrow $u$ is one such that:
    \begin{enumerate}
        \item $f \circ u$ is defined implies $f \circ u = f$
        \item $u \circ g$ is defined implies $u \circ g = g$
    \end{enumerate}
    
    The data of the metacategory $C$ is required to satisfy three axioms:
    \begin{enumerate}
        \item Composite $(h \circ g) \circ f$ is defined iff composite $h \circ (g \circ f)$ is defined. The two composites are equal if both are defined, with the triple composite written as $h \circ g \circ f$.
        \item Triple composite $h \circ g \circ f$ is defined iff both $h \circ g$ and $g \circ f$ are defined.
        \item For each arrow $g$ of category $C$, there exists identity arrows $u$ and $u'$ such that $u' \circ g$ and $g \circ u$ are defined.
    \end{enumerate}
\end{definition}

\begin{remark}
    With these axioms, any metacategory can be defined by either using the objects-and-arrows axioms, or by arrows-only axioms alone -- they are equivalent. Identity arrows in the arrows-only axioms simply substitute for explicit objects in the objects-and-arrows axioms.
\end{remark}

\subsection{Categories}

\begin{definition}[Category]
    A \textit{category} is any interpretation of the category axioms within \textit{set theory} (as opposed to a metacategory).
\end{definition}

\begin{definition}[Directed Graph (Diagram Scheme)]
    A \textit{directed graph} (or \textit{diagram scheme}) is
    \begin{enumerate}
        \item \textbf{Set of objects}: $O$.
        \item \textbf{Set of arrows}: $A$.
        \item Two functions:
        \begin{figure}[H]
            \centering
            \begin{tikzcd}
                A \arrow[r, shift left, "\mathrm{Dom}"] \arrow[r, shift right, swap, "\mathrm{Cod}"] & O
            \end{tikzcd}
            \caption{Category functions.}
            \label{fig:category-functions}
        \end{figure}
    \end{enumerate}
    
    The set of \textit{composable pairs of arrows} is given as the set of \textit{product over $O$} ($\ProductOver{O}$)
    \begin{equation}
        A \ProductOver{O} A \coloneqq \set{ 
            \langle g, f \rangle \given
            g, f \in A \land \Domain{g} = \Codomain{f}
        }
    \end{equation}
\end{definition}

\begin{definition}[Category]
    A \textit{category} is a directed graph with two additional functions
    \begin{enumerate}
        \item \textbf{Identity}.
        \begin{equation}
            \Id \colon O \to A, \quad c \mapsto \Id_c
        \end{equation}
        \item \textbf{Composition}.
        \begin{equation}
            \circ \colon A \ProductOver{O} A \to A, \quad \langle g, f \rangle \mapsto g \circ f
        \end{equation}
    \end{enumerate}
    
    Such that for all objects $a \in O$ and all composable pairs of arrows $\langle g, f \rangle \in A \ProductOver{O} A$, given the associativity and unit axioms hold,
    \begin{gather}
        \Domain{\Id \circ a} = a = \Codomain{\Id \circ a} \\
        \Domain{g \circ f} = \Domain{f} \\
        \Codomain{g \circ f} = \Codomain{g}
    \end{gather}
\end{definition}

\begin{remark}
    In the context of a category $C$, letters $A$ and $O$ for the set of arrows and objects are omitted for brevity without loss of information, then
    \begin{itemize}
        \item $c \in C$ means $c \in O$: $c$ is an object of $C$.
        \item $f \In C$ means $f \in A$: $f$ is an arrow of $C$.
    \end{itemize}
\end{remark}

\begin{definition}[Hom-Set]
    The \textit{hom-set} $\HomSet{b, c}$ is the set of arrows from $b$ to $c$, i.e. arrows $f$ with $\Domain{f} = b$ and $\Codomain{f} = c$.
\end{definition}

\begin{example}
    Some examples of categories include
    \begin{table}[H]
        \centering
        \begin{tabular}{@{} c l @{}}
            \toprule
            Name & Description \\
            \midrule
            \Category{0} & Empty category (0 objects, 0 arrows). \\
            \Category{1} & 1 object, 1 arrow (identity). \begin{tikzcd}[row sep=1em, column sep=1em]
                \bullet \arrow[loop right]
            \end{tikzcd} \\
            \Category{2} & 2 objects, 1 non-identity arrow. 
            \begin{tikzcd}[ampersand replacement=\&, row sep=1em, column sep=1em]
                \bullet \arrow[r] \arrow[loop left] \& \bullet \arrow[loop right]
            \end{tikzcd} \\
            \Category{3} & 3 objects, non-identity arrows. 
            \begin{tikzcd}[ampersand replacement=\&, row sep=1em, column sep=1em]
                \bullet \arrow[r] \arrow[loop left] \arrow[rd] \& \bullet \arrow[d] \arrow[loop right] \\
                \& \bullet \arrow[loop right]
            \end{tikzcd} \\
            $\downdownarrows$ & 2 objects, 2 parallel non-identity arrows $a \rightrightarrows b$. 
            \begin{tikzcd}[ampersand replacement=\&, row sep=1em, column sep=1em]
                \bullet \arrow[r, shift left] \arrow[r, shift right] \arrow[loop left] \& \bullet \arrow[loop right]
            \end{tikzcd} \\
            \bottomrule
        \end{tabular}
        \caption{Example categories.}
        \label{tab:example-categories}
    \end{table}
\end{example}
